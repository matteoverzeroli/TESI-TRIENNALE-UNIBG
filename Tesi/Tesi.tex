\documentclass[
	a4paper,
	cleardoublepage=empty,
	headings=twolinechapter,
	numbers=autoenddot,
]{scrbook}

\usepackage{amsmath}
\usepackage{amsfonts}
\usepackage{amssymb}
\usepackage{import}
\usepackage{float}
\usepackage{cite}
\usepackage[signatures,swapnames,sans]{frontespizio}

\usepackage{todonotes}
\usepackage{verbatim}

\usepackage{color}
\usepackage{url} 

\newcommand{\Fig}[0]{Fig.}
\newcommand{\Eq}[0]{Eq.}

\pagestyle{plain}
% Language(s)
\usepackage[italian]{babel}

% Fonts & typography
\usepackage[T1]{fontenc}
\usepackage[utf8]{inputenc}
\usepackage{microtype,mathtools,amssymb}
\usepackage{lmodern}

% Margins
\usepackage[top=3cm,bottom=3cm,inner=3cm,outer=3cm]{geometry}
\usepackage{multicol}
\usepackage{multirow}

% Formulas
\usepackage{physics,siunitx,xfrac}
\def\diffd{\textnormal{d}}
\sisetup{
  detect-all,
  binary-units,
  per-mode=symbol,
% per-mode=symbol-or-fraction,
% fraction-function=\sfrac,
% inter-unit-product=\ensuremath{{}\cdot{}},
  retain-unity-mantissa=false,
  retain-explicit-plus=true,
  exponent-product=\cdot,
  complex-root-position=before-number,
  output-complex-root=\ensuremath{\mathrm{j}},
  separate-uncertainty,
}

% Tabulars
\usepackage{array,booktabs}

% Images & floats
\usepackage{graphicx,adjustbox}
\usepackage{caption,subcaption}
\graphicspath{{img/}}
\captionsetup{
  tableposition=top,
  figureposition=bottom,
  width=0.9\textwidth,
  format=plain,
  indent=0mm,%5mm,
  labelformat=simple,
  labelsep=endash,
  font=small,
  labelfont={sf,bf},
}
\captionsetup[sub]{width=0.8\linewidth}

% Cross-references & links
\usepackage{hyperref,cleveref,bookmark}
\hypersetup{
  unicode=true,
  breaklinks=true,
  bookmarks=true,
  bookmarksnumbered=true,
  pdftitle={},
  pdfauthor={},
}
\urlstyle{same}

% Extra
% \renewcommand{\thefootnote}{\Roman{footnote}}

\renewcommand*{\vec}[1]{\ensuremath{\underline{#1}}}
\newcommand*{\mat}[1]{\ensuremath{\mathbf{#1}}}
\newcommand*{\tran}{^\mathsf{T}}

\AtBeginDocument{%
  \renewcommand{\epsilon}{\varepsilon}
  \renewcommand{\theta}{\vartheta}
  \renewcommand{\kappa}{\varkappa}
% \renewcommand{\rho}{\varrho}
  \renewcommand{\phi}{\varphi}
}

% Alter some LaTeX defaults for better treatment of floats
\renewcommand{\topfraction}{0.9}
\renewcommand{\bottomfraction}{0.8}
\setcounter{topnumber}{2}
\setcounter{bottomnumber}{2}
\setcounter{totalnumber}{4}
\setcounter{dbltopnumber}{2}
\renewcommand{\dbltopfraction}{0.9}
\renewcommand{\textfraction}{0.07}
\renewcommand{\floatpagefraction}{0.7}
\renewcommand{\dblfloatpagefraction}{0.7}




\newcolumntype{C}{>{\centering\arraybackslash}p{3cm}}

\begin{document}
	\frontmatter
	
	\pdfbookmark{Title page}{titlepage}
	\begin{frontespizio}
		\Margini{3cm}{3cm}{3cm}{3cm}
		\Universita{Bergamo}
		\Logo[43.332mm]{ImageFiles/unibg-mark}
		\Divisione{Scuola di Ingegneria}
		\Corso[Laurea Triennale]{Ingegneria Informatica\\Classe n. L-8 Ingegneria dell’informazione (D.M. 270/04)}
		\Titolo{Progetto di un sistema optoelettronico indossabile per misure fotopletismografiche}
		\Punteggiatura{}
		\NCandidato{Tesi di Laurea Triennale}
		\Candidato[1060652]{Wasim Essbai}
		\Candidato[1057926]{Matteo Verzeroli}
		\Relatore{Prof.\ Gianluca Traversi}
		\Correlatore{Dott.\ Andrea Pedrana}
		\Annoaccademico{2020--2021}
		\begin{Preambolo*}
			\usepackage[italian]{babel}
			\usepackage[T1]{fontenc}
			\usepackage[utf8]{inputenc}
			\usepackage{microtype}
			\usepackage{lmodern}
			\graphicspath{{img/}}
			
			\renewcommand{\frontinstitutionfont}{\fontsize{14}{17}\bfseries\scshape}
			\renewcommand{\fronttitlefont}{\fontsize{17}{21}\bfseries\scshape}
			\renewcommand{\frontfootfont}{\fontsize{12}{14}\bfseries\scshape}
		\end{Preambolo*}
	\end{frontespizio}
	
	\tableofcontents
	\mainmatter
	
	\chapter*{Introduzione}
	\addcontentsline{toc}{chapter}{Introduzione}
	
	\chapter{Fotopletismografia}
	\import{./TextFiles/Fotopletismografia/}{Principio di funzionamento.tex}
	\import{./TextFiles/Fotopletismografia/}{Siti di misura e lunghezze d’onda impiegate.tex}
	\import{./TextFiles/Fotopletismografia/}{Stato dell’arte dei moduli PPG.tex}
	
	\chapter{Progetto della piattaforma indossabile}
	\import{./TextFiles/Progetto della piattaforma indossabile/}{Hardware dell’adapter board.tex}
	\import{./TextFiles/Progetto della piattaforma indossabile/}{Firmware.tex}
	
	\chapter{Caratterizzazione della piattaforma indossabile}
	\import{./TextFiles/Caratterizzazione della piattaforma indossabile/}{Consumo di potenza.tex}
	\import{./TextFiles/Caratterizzazione della piattaforma indossabile/}{Sito di misura.tex}
	
	\chapter*{Conclusioni}
	\addcontentsline{toc}{chapter}{Conclusioni}
	
	\bibliographystyle{unsrt}
	\bibliography{./OtherFiles/Bibliography_Fotopletismografia,./OtherFiles/Bibliography_DatasheetReferences,./OtherFiles/Bibliography_Hardware}
	
	\addcontentsline{toc}{chapter}{Bibliografia}
	
\end{document}