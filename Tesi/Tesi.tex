\documentclass[
	a4paper,
	cleardoublepage=empty,
	headings=twolinechapter,
	numbers=autoenddot,
]{scrbook}

\usepackage{amsmath}
\usepackage{amsfonts}
\usepackage{amssymb}
\usepackage{import}
\usepackage{float}
\usepackage{cite}
\usepackage[signatures,swapnames,sans]{frontespizio}

\usepackage{todonotes}
\usepackage{verbatim}

\usepackage{color}
\usepackage{url} 

\newcommand{\Fig}[0]{Fig.}
\newcommand{\Eq}[0]{Eq.}

\pagestyle{plain}
% Language(s)
\usepackage[italian]{babel}

% Fonts & typography
\usepackage[T1]{fontenc}
\usepackage[utf8]{inputenc}
\usepackage{microtype,mathtools,amssymb}
\usepackage{lmodern}

% Margins
\usepackage[top=3cm,bottom=3cm,inner=3cm,outer=3cm]{geometry}
\usepackage{multicol}
\usepackage{multirow}

% Formulas
\usepackage{physics,siunitx,xfrac}
\def\diffd{\textnormal{d}}
\sisetup{
  detect-all,
  binary-units,
  per-mode=symbol,
% per-mode=symbol-or-fraction,
% fraction-function=\sfrac,
% inter-unit-product=\ensuremath{{}\cdot{}},
  retain-unity-mantissa=false,
  retain-explicit-plus=true,
  exponent-product=\cdot,
  complex-root-position=before-number,
  output-complex-root=\ensuremath{\mathrm{j}},
  separate-uncertainty,
}

% Tabulars
\usepackage{array,booktabs}

% Images & floats
\usepackage{graphicx,adjustbox}
\usepackage{caption,subcaption}
\graphicspath{{img/}}
\captionsetup{
  tableposition=top,
  figureposition=bottom,
  width=0.9\textwidth,
  format=plain,
  indent=0mm,%5mm,
  labelformat=simple,
  labelsep=endash,
  font=small,
  labelfont={sf,bf},
}
\captionsetup[sub]{width=0.8\linewidth}

% Cross-references & links
\usepackage{hyperref,cleveref,bookmark}
\hypersetup{
  unicode=true,
  breaklinks=true,
  bookmarks=true,
  bookmarksnumbered=true,
  pdftitle={},
  pdfauthor={},
}
\urlstyle{same}

% Extra
% \renewcommand{\thefootnote}{\Roman{footnote}}

\renewcommand*{\vec}[1]{\ensuremath{\underline{#1}}}
\newcommand*{\mat}[1]{\ensuremath{\mathbf{#1}}}
\newcommand*{\tran}{^\mathsf{T}}

\AtBeginDocument{%
  \renewcommand{\epsilon}{\varepsilon}
  \renewcommand{\theta}{\vartheta}
  \renewcommand{\kappa}{\varkappa}
% \renewcommand{\rho}{\varrho}
  \renewcommand{\phi}{\varphi}
}

% Alter some LaTeX defaults for better treatment of floats
\renewcommand{\topfraction}{0.9}
\renewcommand{\bottomfraction}{0.8}
\setcounter{topnumber}{2}
\setcounter{bottomnumber}{2}
\setcounter{totalnumber}{4}
\setcounter{dbltopnumber}{2}
\renewcommand{\dbltopfraction}{0.9}
\renewcommand{\textfraction}{0.07}
\renewcommand{\floatpagefraction}{0.7}
\renewcommand{\dblfloatpagefraction}{0.7}




\newcolumntype{C}{>{\centering\arraybackslash}p{3cm}}

\begin{document}
	\frontmatter
	
	\pdfbookmark{Title page}{titlepage}
	\begin{frontespizio}
		\Margini{3cm}{3cm}{3cm}{3cm}
		\Universita{Bergamo}
		\Logo[43.332mm]{ImageFiles/unibg-mark}
		\Divisione{Scuola di Ingegneria}
		\Corso[Laurea Triennale]{Ingegneria Informatica\\Classe n. L-8 Ingegneria dell’informazione (D.M. 270/04)}
		\Titolo{Progetto di un sistema optoelettronico indossabile per misure fotopletismografiche}
		\Punteggiatura{}
		\NCandidato{Tesi di Laurea Triennale}
		\Candidato[1060652]{Wasim Essbai}
		\Candidato[1057926]{Matteo Verzeroli}
		\Relatore{Prof.\ Gianluca Traversi}
		\Correlatore{Dott.\ Andrea Pedrana}
		\Annoaccademico{2020--2021}
		\begin{Preambolo*}
			\usepackage[italian]{babel}
			\usepackage[T1]{fontenc}
			\usepackage[utf8]{inputenc}
			\usepackage{microtype}
			\usepackage{lmodern}
			\graphicspath{{img/}}
			
			\renewcommand{\frontinstitutionfont}{\fontsize{14}{17}\bfseries\scshape}
			\renewcommand{\fronttitlefont}{\fontsize{17}{21}\bfseries\scshape}
			\renewcommand{\frontfootfont}{\fontsize{12}{14}\bfseries\scshape}
		\end{Preambolo*}
	\end{frontespizio}
	
	\tableofcontents
	\listoffigures
	\mainmatter
	
	\chapter*{Introduzione}
	\addcontentsline{toc}{chapter}{Introduzione}
	Il presente lavoro di tesi, svolto presso il Laboratorio di Microelettronica dell’Università degli Studi di Bergamo, tratta la progettazione e lo sviluppo di un sistema optoelettronico indossabile per misure fotopletismografiche. In particolare l'attività si è focalizzata sul progetto di due \textit{Adapter Board} dalle dimensioni ridotte. Questi sistemi stanno acquisendo sempre più popolarità dal momento che permettono un monitoraggio non invasivo dei principali parametri fisiologici delle persone. Infatti, il monitoraggio continuo permette di prevenire alcune patologie, migliorando la qualità della vita. I dispositivi basati su fotopletismografia sono degli ottimi candidati per l'utilizzo nel monitoraggio continuo, dal momento che possono essere integrati in diversi accessori utilizzati nella vita quotidiana, come orologi ed auricolari.
	
	Nel \textit{primo capitolo} viene descritto l'apparato circolatorio e le dinamiche che ne regolano il funzionamento. Si tratta di un aspetto importante che permette di comprendere il principio di funzionamento della fotopletismografia che viene presentato nel capitolo, descrivendo i fattori che influenzano le acquisizioni fotopletismografiche e i principali siti di misura impiegati. Infine, vengono riportati alcuni dispositivi per l'acquisizione di segnali PPG presentati come stato dell'arte.
	
	Nel \textit{secondo capitolo} viene presentato il progetto delle due piattaforme. In particolare si descrive lo sviluppo dei due sistemi progettati e la piattaforma \textit{STM32F4DISCOVERY}, prodotta da STMicroelectronics, utilizzata come scheda di valutazione. Successivamente, viene descritto il firmware sviluppato per le due \textit{Adapter Board} e la macchina a stati finiti che ne descrive formalmente il funzionamento. In conclusione, sono riportati i risultati di alcune misure effettuate, su due soggetti, utilizzando i sistemi realizzati, commentando la qualità del segnale ottenuto. Nel capitolo vengono riportate anche le configurazioni impiegate per le due \textit{Adapter Board} nell'acquisizione dei segnali PPG.
	
	\chapter{Fotopletismografia}
	\import{./TextFiles/Fotopletismografia/}{Principio di funzionamento.tex}
	\import{./TextFiles/Fotopletismografia/}{Siti di misura e lunghezze d’onda impiegate.tex}
	\import{./TextFiles/Fotopletismografia/}{Stato dell’arte dei moduli PPG.tex}
	
	\chapter{Progetto della piattaforma indossabile}
	Lo studio presentato in questo capitolo è l'evoluzione di un progetto di tesi precedentemente svolto presso il laboratorio di elettronica dell’Università degli Studi di Bergamo \cite{Ingegneria2018}. Tuttavia, si è cercato di inserire nuovi elementi per migliorare la board precedentemente progettata, ponendo l'attenzione su due principali aspetti innovativi: lo studio della luce blu, che tipicamente non è utilizzata per misure fotopletismografiche e la riduzione delle dimensioni delle board. In particolare, il sensore MAX86916 permetterà di analizzare i segnali PPG generati da luce rossa, infrarossa, verde e blu, mentre le dimensioni ridotte del sensore MAXM86161 permetteranno di ridurre la grandezza dell'\textit{Adpter Board} progettata e potenzialmente migliorare le prestazioni ottenute con il sensore MAX30101, analizzato nel lavoro precedente\cite{Ingegneria2018}.
	
	\import{./TextFiles/Progetto della piattaforma indossabile/}{Hardware dell’adapter board.tex}
	\import{./TextFiles/Progetto della piattaforma indossabile/}{Firmware.tex}
	\import{./TextFiles/Progetto della piattaforma indossabile/}{Macchina a stati finiti.tex}
	\import{./TextFiles/Progetto della piattaforma indossabile/}{Misure preliminari.tex}
	
	\chapter*{Conclusioni}
	\addcontentsline{toc}{chapter}{Conclusioni}
	Questa attività di tesi, svolta presso il laboratorio di elettronica dell'Università degli Studi di Bergamo, ha avuto come obiettivo la progettazione e la realizzazione di due schede prototipali per effettuare misure fotopletismografiche. Le due PCB progettate si differenziano dai sistemi già realizzati nei lavori precedenti e si propongono come una loro evoluzione. In particolare, si è cercato di minimizzare il più possibile le dimensioni delle \textit{Adapter Board}. Ciò è stato reso possibile grazie all'utilizzo del sensore MAXM86161 il quale, integrando un LDO, ha permesso di ridurre i componenti necessari. La miniaturizzazione dei sistemi elettronici indossabili è diventata ormai una prerogativa nel campo dei sensori pensati per un monitoraggio continuo dei parametri fisiologici della persona. Nell'ultimo decennio, l'interesse verso lo studio dei segnali PPG è fortemente aumentato, grazie alla diffusione di smartwatch, dispositivi per il monitoraggio dell'attività sportiva e per applicazioni in ambito medicale. Ad oggi, l'elettronica necessaria per questa tecnologia ha raggiunto dimensioni tali da poter essere integrata in qualsiasi supporto, come ad esempio orologi, fasce e auricolari. Inoltre, i consumi dei sensori integrati sono tali da garantire un'autonomia sufficiente anche in dispositivi alimentati a batteria. Infine, a differenza di altre tecnologie che permettono il monitoraggio di parametri fisiologici, i dispositivi PPG sono poco invasivi, pur mantenendo una buona accuratezza.
	
	Al contrario, la caratteristica peculiare del sensore MAX86916 è l'integrazione di un LED di colore blu, che non è mai stato studiato in lavori precedenti. Questa board permetterà di approfondire la qualità dei segnali ottenuti con una lunghezza d'onda di \SI{460}{\nano\meter}. Tuttavia, questo lavoro di tesi non si concentra sulla caratterizzazione dei segnali e la valutazione approfondita della loro qualità. Infatti, saranno necessari studi successivi per determinare in modo preciso e quantitativo le performance dei sistemi realizzati. Sarà inoltre necessario validare i risultati tramite il confronto con dei tracciati ECG, che permettono di ottenere misure molto precise e accurate del ritmo cardiaco.
	
	Osservando i risultati ottenuti nel paragrafo \ref{cap:misure_preliminari} sui due soggetti analizzati, si può concludere che il sensore MAX86916 permette delle acquisizioni di qualità superiore rispetto al modulo MAXM86161 in tutti i siti di misura analizzati. Confrontando i tracciati ottenuti, la luce verde sembra aver prodotto i segnali migliori in tutti i siti su entrambi i soggetti, sebbene nelle acquisizioni sul soggetto 1 la luce infrarossa sia risultata di ampiezza maggiore nelle misure sul lobo, polso e sulla fronte. In generale, i LED rosso e infrarosso sembrano essere più sensibili ai movimenti, seppur minimi, dei soggetti. In particolare, i LED rosso e infrarosso hanno prodotto un segnale rumoroso nelle acquisizioni sulla parte antero-interna del polso, più soggetta a movimenti, anche involontari. Al contrario, i segnali del LED verde e blu sembrano essere meno soggetti ai disturbi, generando dei segnali molto puliti. Anche le acquisizioni sul lobo e sulla fronte sono risultate ottime, in accordo con i risultati analizzati in letteratura.
	
	I risultati rilevati in questo progetto non sono da considerarsi definitivi. Infatti, sarà necessario effettuare delle acquisizioni su un campione di soggetti maggiore, diversificati per età, sesso, condizioni ambientali in cui si svolgono le misure, colore della carnagione, condizione del soggetto (a riposo, sotto sforzo, in movimento) e analizzare degli indici quantitativi (come l'indice di perfusione) per analizzare la qualità dei segnali. Un'ulteriore sviluppo futuro sarà l'introduzione dell'accelerometro, il cui utilizzo è stato progettato ma non è stato montato sulle schede. Infine, dovrà essere anche valutato il rapporto tra prestazioni e consumo, in modo da configurare opportunamente i sensori PPG, utilizzando anche i filtri presenti \textit{on board} che dovrebbero migliorare la qualità delle misure.
	

	\bibliographystyle{unsrt}
	\bibliography{./OtherFiles/Bibliography_Fotopletismografia,./OtherFiles/Bibliography_DatasheetReferences,./OtherFiles/Bibliography_Hardware,./OtherFiles/Bibliography_Firmware,./OtherFiles/Bibliography_Misure_Preliminari}
	
	\addcontentsline{toc}{chapter}{Bibliografia}
	
\end{document}