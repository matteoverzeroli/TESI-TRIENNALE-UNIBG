\section{Siti di misura e lunghezze d'onda impiegate}
\todo{sistemare riferimenti bibliografia (preferibilmente no treccani)}
Il fenomeno dell'assorbimento rappresenta una riduzione dell'energia luminosa.
Da una pubblicazione sul Journal of Biomedical Optics sulle proprietà ottiche della pelle umana si evince che nei tessuti umani questo è dovuto principalmente a due sostanze: l'emoglobina e la melanina[Optical proprieties of human skin].
L'emoglobina, spesso indicata con la sigla Hb, è una proteina contenente ferro in grado di combinarsi in modo reversibile con l’ossigeno molecolare [Treccani]. Essa presenta, all'interno nella regione della luce visibile, tre picchi di assorbimento. Il primo (noto anche come Picco di Soret), che è quello dominante, si riferisce alla regione blu dello spettro, gli altri due invece si possono distinguere nella regione giallo-verde, avente lunghezza d'onda tra i 500 e 600 nm. Questi tre picchi combinati danno all'emoglobina un colore rosso.
La melanina è contenuta sotto forma di granuli nelle cellule dello strato basale dell’epidermide [Treccani]. Questa assorbe principalmente le lunghezze d'onda più corte, quindi uno spettro di assorbimento che decresce gradualmente dall'ultravioletto fino all'infrarosso. In realtà la melanina presenta una complessità elevata; infatti, la sua struttura non è ancora bene nota, ed è in corso di studio e oggetto di dibattito scientifico.
Un altro elemento di cui tener conto è l'acqua che in realtà presenta una basso assorbimento nella regione visibile, mentre assorbe la luce nel regime ultravioletto e del lontano infrarosso. Le regioni di luce che riescono ad attraversare maggiormente i tessuti umani sono quindi quella rossa e del vicino infrarosso, che le rende fonti utilizzabili per applicazioni di fotopletismografia.
La profondità nel tessuto umano che si può raggiungere dipende dalla lunghezza d'onda ma anche dalla distanza tra la sorgente luminosa e il foto-rilevatore. Le sorgenti rossa e infrarossa permettono acquisizioni a maggiore profondità nel tessuto. La luce verde invece, per via dell'emoglobina e della deossiemoglobina, viene assorbita in quantità superiore a quella infrarossa \cite{Lee2021}. Infatti, nelle zone in cui il sangue pulsa attraverso la pelle si rileva una variazione tra luce emessa e riflessa maggiore per la regione verde rispetto a quella infrarossa. Questa caratteristica per mettere di avere un rapporto segnale-rumore (Signal to Noise Ratio), per rilevazioni con luce verde, superiore a quello ottenibile con luce infrarossa. Pertanto, la luce verde risulta più adatta per misure superficiali come la variazione del flusso sanguigno superficiale \cite{Youssef2020}.

 